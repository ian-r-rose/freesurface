\subsection{BulkModulus}

Bulk Modulus K is the resistance to volume compression, units is Pa \\
$K=-V\frac{\partial P}{\partial V}$ \\
$K_{s}$ is de adiabatic bulk modulus (written as $K_{s} = \gamma P$ (gamma is not Grunneissen parameter in this case but the adiabatic index, just to confuse us) \\
typical values of K \\
water $2.2\times 10^{9}$ solid $5 \times 10^{7}$ Pa \\
$1/K$ is called compressibility \\
in the Tan and Gurniss paper they use the ratio $Di/\gamma$ for the compressibility which is correct since
$\gamma = \frac{\alpha K}{c_{p} \rho}$ and $Di = \frac{\alpha g h}{c_{p}}$ which results in, \\
$\frac{Di}{\gamma} = \frac{g h \rho_{1}}{K}$ with $K = g h \rho_{2}$ \\
$Pa = kg/m s^{2} = \rho g h = (kg/m^{3}) (m/s^{2}) (m)$ \\
this means that (if we take g and h constant) we have the change in density over the model domain $\frac{\rho_{1}}{\rho_{2}}$.

